% Copyright (c) 2023-present, Adam McKellar & Kai Rothe
% All rights reserved.
%
% This source code is licensed under the license found in the
% LICENSE file in the root directory of this source tree.

% !TEX engine = luatex
% !TeX spellcheck = en_EN


\documentclass{beamer}

\usepackage{amsmath}
\usepackage{amssymb}

\usepackage{hyperref}
\hypersetup{
	colorlinks,
	allcolors=.,
	urlcolor=blue,
}

\usepackage{fontspec}
\usepackage{polyglossia}
\setmainlanguage{english}

\usepackage{showexpl}

\usepackage[useregional]{datetime2}

% https://hartwork.org/beamer-theme-matrix/
\usetheme{Antibes}
\usecolortheme{beaver}
\useinnertheme{circles}

\usepackage[
	type={CC},
	modifier={by},
	version={4.0},
	imagemodifier={-80x15},
]{doclicense}

\author{Adam McKellar\qquad Kai Rothe\qquad}
\title{Feedback Arc Set in Tournaments}
\date{\today}

\newcommand{\abs}[1]{\| #1 \|}

\begin{document}
	\frame{\titlepage}
	
	\begin{frame}{Overview}
		\tableofcontents
		\tiny
		\doclicenseThis
	\end{frame}

	\section{Feedback Arc Set in Tournaments}
	\begin{frame}[fragile]{Feedback Arc Set (FAS)}
		\begin{align*}			
			FAS := \{&(G = (V, E), k) | G \text{ is directed } \\
										&\land \exists E' \subset E : \abs{E'} \leq k : G' = (V, E - E') \text{ is acyclic}  \}		
		\end{align*}
	\end{frame}
	\begin{frame}[fragile]{Feedback Arc Set (FAS) Example}
		Example \(G\) :
		\begin{center}
			\includegraphics<1>[height=0.3\paperheight]{images/FAS/cyclic_graph_example.pdf}
			\includegraphics<2>[height=0.3\paperheight]{images/FAS/cyclic_graph_example_highlight_cycles.pdf}
			\includegraphics<3>[height=0.3\paperheight]{images/FAS/cyclic_graph_example_highlight_solution_k2.pdf}
			\includegraphics<4->[height=0.3\paperheight]{images/FAS/cyclic_graph_example_highlight_solution_k1.pdf}
		\end{center}
		\begin{itemize}
			\item<3-> Solvable for \(k=2\)
			\item<4-> Solvable for \(k=1\)
			\item<5-> \(\Rightarrow (G, 1) \in FAS\)
		\end{itemize}
	\end{frame}

	\begin{frame}[fragile]{Tournaments (T)}
		\begin{center}
			\includegraphics<1>[height=0.3\paperheight]{images/T/complete_graph_example.pdf}
			\includegraphics<2->[height=0.3\paperheight]{images/T/tournament_example.pdf}
		\end{center}
		\begin{itemize}
			\item<1-> Take a complete graph.
			\item<2-> Assign every edge a direction.
			\item<3-> \(T := \{ (G = (V, E)) | \forall u, v \in V : (u, v) \in E \veebar (v, u) \in E \}\)
		\end{itemize}
	\end{frame}
	\begin{frame}[fragile]{Feedback Arc Set in Tournaments (FAST)}
		\[	FAST := \{ (G = (V, E), k) | G \in T \land (G, k) \in FAS \} \]
		\begin{center}
			\includegraphics<2>[height=0.3\paperheight]{images/T/tournament_example.pdf}
			\includegraphics<3->[height=0.3\paperheight]{images/FAST/fast_example.pdf}
		\end{center}
		\begin{itemize}
			\item<3-> \((G, 1)\in FAST\)
		\end{itemize}
	\end{frame}

	\section{Lemma 2.5}
	\begin{frame}[fragile]{Lemma 2.5}
		\begin{itemize}[<+->]
			\item first line
		\end{itemize}
	\end{frame}
	%TODO

	\section{Observation 2.6}
	%TODO
	
	\section{Lemma 2.7}
	%TODO
	
	\section{Lemma 2.7: minimum feedback arc sets coincide with minimum inversion sets}
	%TODO
	
	\section{Lemma 2.7: two reduction rules for feedback arc set in tournaments}
	%TODO
	
	\section{Theorem 2.8: quadratic kernel for feedback arc sets in tournaments}
	%TODO
	
\end{document}